\documentclass{siamltex}

\usepackage{amsmath}
\usepackage{amssymb}
\usepackage{algorithm}
\usepackage{algorithmic}
\usepackage{hyperref}


% Alias
\newcommand*{\beq}{\begin{equation}}
\newcommand*{\eeq}{\end{equation}}
\newcommand*{\bseq}{\begin{subequations}}
\newcommand*{\eseq}{\end{subequations}}
\newcommand*{\bi}{\begin{itemize}}
\newcommand*{\ei}{\end{itemize}}
\newcommand*{\be}{\begin{enumerate}}
\newcommand*{\ee}{\end{enumerate}}

\newcommand*{\cP}{\mathcal P}

\newcommand*{\eqnok}[1]{(\ref{#1})}

\DeclareMathOperator*{\st}{\text{s.t.}}
\DeclareMathOperator*{\half}{\frac{1}{2}}

\DeclareMathOperator*{\bR}{{\mathbb R}}

\title{Projection onto a simplex\footnote{\today}}
\author{ }

\begin{document}
\maketitle

\begin{abstract} Explains how to find a projection onto a simplex.
\end{abstract}
%
%\begin{keywords}
%\end{keywords}
%
%\begin{AMS}
%\end{AMS}

\section{Projection onto a face($n-1$ simplex)}\label{sec:proj.face}
Consider a projection problem as below:
\bseq\begin{alignat}{2}\label{eq:proj.face}
 \min_{x} &~~ \half\|x-z\|_2^2 \\
 \st      &~~ e^T x ~=~ b \label{eq:proj.face.eq}\\
          &~~ x ~\ge~ 0.
\end{alignat}\eseq
where $e\in\bR^n$ is a vector of all $1$'s.
Then the KKT condition of \eqnok{eq:proj.face} is
\begin{alignat}{2}
 0 ~\le~ x-z+\lambda e & ~\perp~ &~ x ~\ge~ 0
\end{alignat}
where $\lambda$ is the Largrange multiplier to \eqnok{eq:proj.face.eq}. 
So the solution $x^*$ of \eqnok{eq:proj.face} is
\beq
 x^* ~=~ (z-\lambda e)_+ ~=~ \max\{z-\lambda e,~ 0\} 
\eeq
where the $\max$ operator applies element-wise. If we define a function $g(\lambda)$ like below 
\beq
 g(t) ~:=~ \sum_{i:z_i-\lambda\ge0} (z_i-\lambda) ~=~ \sum_{i=1}^n x_i ~=~ e^Tx
\eeq
then the optimal solution $x^*$ of \eqnok{eq:proj.face} can be obtained by finding a value of $\lambda^*$ such that $g(\lambda^*)=b$.

Without loss of generality, we assume that the vector $z$ is sorted in descending order. Let $w$ be a vector such that
\beq
 w_k ~=~ \sum_{i=1}^k z_i.
\eeq
Assume that $\lambda_k$ is the solution of $g(\lambda)=b$ when the first $k$ entries of $z$ have $z_i-\lambda^k\ge0$. Then from 
\beq
 \sum_{i=1}^k(z_i-\lambda_k) ~=~ b
\eeq
we have
\beq
 \lambda_k ~=~ \frac{w_k-b}{k}.
\eeq
And we need to find the $k^*$ such that $z_{k^*}-\lambda_{k^*}\ge0$ and $z_{k^*+1}-\lambda_k^*\le0$. Then $\lambda^*=\lambda_{k^*}$ and $x^*=(z-\lambda^*e)_+$.

\section{Projection onto $n$-simplex}
Now we consider a projection onto $n$-simplex:
\bseq\begin{alignat}{2}\label{eq:proj.simplex}
 \min_{x} &~~ \half\|x-z\|_2^2 \\
 \st      &~~ e^T x ~\le~ b \label{eq:proj.simplex.ineq}\\
          &~~ x ~\ge~ 0.
\end{alignat}\eseq
The KKT condition of \eqnok{eq:proj.simplex} is
\bseq\begin{alignat}{2}
 0 ~\le~ x-z+\lambda e  & ~\perp~ &~ x ~\ge~ 0 \\
 0 ~\le~ b-e^T x & ~\perp~ &~ \lambda ~\ge~ 0&.
\end{alignat}\eseq
Again, $\lambda$ is the Lagrange multiplier.

First, if $e^Tz_+\le b$, then it can be easily shown that $x^* = z_+$ is a solution. Thus assume that $e^Tz_+>b$. Then the solution should be $x^*=(z-\lambda^* e)_+$ from the KKT condition where $\lambda^*$ is the optimal Lagrange multiplier. Also we can see that $e^Tx^*=b$. If not, i.e. $e^Tx^*<b$, we should have $\lambda^*=0$ and 
\beq
 b ~>~ e^Tx^* ~=~ e^T(z-\lambda^*e)_+ ~=~ e^Tz_+ ~>~b
\eeq
which contradicts. Thus we can use the same technique discussed in Section \ref{sec:proj.face}. Note that since $e^Tz_+>b$, $\lambda^*$ will be nonnegative if the smallest $k^*$ is chosen, and thus satisfies the KKT condition.

\section{Projection onto an intersection of $n$-simplex and box constraints}
We consider another problem like below:
\bseq\begin{alignat}{2}\label{eq:proj.intersect}
 \min_{x} &~~ \half\|x-z\|_2^2 \\
 \st      &~~ e^T x ~\le~ b \label{eq:proj.intersect.ineq}\\
          &~~ 0 ~\le~ x ~\le~ u.\label{eq:proj.intersect.box}
\end{alignat}\eseq
Since the constraints are convex, we can solve this problem using Dykstra's projection algorithm. Let $C=\{x ~|~ e^Tx\le b\}$ and $D=\{x ~|~ 0\le x\le u\}$. Then the Dykstra's projection algorithm is described in algorithm \ref{alg:Dykstra}.\footnote{The algorithm is from the wiki page. \url{http://en.wikipedia.org/wiki/Dykstra's_projection_algorithm}} To use the algorithm, we need to know the projection onto each convex set $C$ and $D$. For \eqnok{eq:proj.intersect.ineq}, we can use the projection described in previous sections. Since \eqnok{eq:proj.intersect.box} is a simplex box constraint, we can simply truncate to project onto this set.
\begin{algorithm}[h]\small
\caption{$\mbox{\sc Projection onto $C\cap D$}$}\label{alg:Dykstra}
\begin{algorithmic}[1]
\REQUIRE $z$: A point projected.\\
$C, D$: Convex Sets.\\
$\cP_C(x), \cP_D(x)$: Projection functions onto $C$ and $D$, respectively.
\ENSURE $x$: Projection of $z$ onto $C\cap D$.
\medskip
\STATE $k\gets 0$
\STATE $x_0 \gets z$, $y_0 \gets 0$, $p_0 \gets 0$, $q_0 \gets 0$
\medskip
\REPEAT
\STATE $y_k\gets\cP_D(x_k+p_k)$
\STATE $p_{k+1}\gets (x_k-y_k)+p_k$
\STATE $x_{k+1}\gets \cP_C(y_k+q_k)$
\STATE $q_{k+1}\gets (y_k-x_{k+1})+q_k$
\STATE $k\gets k+1$ 
\UNTIL{$x_k, y_k, p_k$ and $q_k$ are fixed points.}
\end{algorithmic}
\end{algorithm}


\end{document}